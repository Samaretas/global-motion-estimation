\section{Conclusions}
\label{sec:conclusions}

This project addresses the task of global motion estimation through the use of an indirect method for the estimation of apparent motion which is based on the affine motion model.

During the implementation of the tool, we built the following scripts:
\begin{itemize}
    \item \texttt{bbme.py} as Python module containing all the functions related to the various block matching motion estimation procedures that we tried;
    \item \texttt{motion.py} as Python module containing all the functions (and constants) needed in order to perform camera motion estimation and motion compensation;
    \item \texttt{results.py} as example of use of the packages modules to produce the results we used also to create this report.  
\end{itemize}

The approach we used proved to be efficient and robust, as long as the motion in the image is not too complex and as long as the background is not uniform and, therefore, the BBME algorithms are able to spot its motion.

The actual implementation is to be found in the following GitHub repository \url{https://github.com/Samaretas/global-motion-estimation}, where the reader can find all the commented scripts and try out the code.
In the repository it will be possible to run the basic example with \texttt{pan240.mp4}, whether the other videos of the dataset can be found here \url{https://drive.google.com/drive/folders/1gZisWe4DEWpb_CoHkTi6OKnxhl5Ca_mT?usp=sharing}.

For more information on the code and how to perform a full-cycle execution of the pipelen, please refer to the Github repository.