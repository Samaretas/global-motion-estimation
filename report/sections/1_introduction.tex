\section{Introduction}
The problem of global motion estimation (GME) deals with the separation, in a video sequence, of two different types of motion: the egomotion of the camera recording the video, and the actual motion of the objects recorded. This task is crucial for a number of application, for instance:
\begin{itemize}
    \item removal of camera shaking or noisy motion;
    \item study of the motion of the camera itself;
    \item object segmentation and recognition;
    \item object tracking;
\end{itemize}

For these reasons, motion estimation is a task that has long been studied; in the literature we can find a variety of approaches to the problem. In this work we concentrated on some of the most effective approaches and combined them to get an indirect, multi-resolution and robust approach to GME.

The document is structured as follows: in section [...] we sum up the fundamental aspects of the GME problem and the standard approaches to solve these problems. In section [...] we discuss the strategies we chose for our implementation, and finally in section [...] we present the results of this work. 
